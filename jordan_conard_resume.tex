%%%%%
% Jordan Conard
% Resume LaTeX Source
% jordan.conard3@gmail.com
% github.com/jconard3/resume
%%%%%

\documentclass[letterpaper]{article}
\usepackage{enumitem}

%-----Dimensions-----%
\usepackage[centering]{geometry}
\geometry{margin=1in}

%-----Document-----%
\begin{document}
\pagestyle{empty} % Remove page numbering
%\setlist{nolistsep} % Remove list separation


%-----Name and Contact-----%
\begin{center}
    {\Huge Jordan Conard}\\~\\
    jordan.conard3@gmail.com\\
\end{center}

%-----Objective-----%
%\section*{Objective}
%Looking for new opportunities at the intersection of DevOps and Security where I can help lead a team to greatness and get stuff done. %I feel gross reading this

%-----Work Experience-----%
\section*{Work Experience}
\begin{tabular}{l|l}
{May '20 - \emph{Current}} & Senior Detection Engineer - Spotify, New York City\\
\end{tabular}
\begin{itemize}[itemsep=0.5pt]
	\item Lead the Security Incident Response program operating a 24/7/365 rotation across U.S. and European time-zones consisting of over 30 responders.
	\item Developed security threat detections to more quickly identify malicious behavior and alert notification pipelines which prompted users for confirmation to reduce the likelihood of a false positive alert.
	\item Created a Blue Team 'Capture The Flag' around an example intrusion into a GCP GKE cluster as a Kubernetes and investigation training exercise for two dozen Security Responders.
	\item Performed due diligence for the Chartable acquisition in coordination with Corporate and Infrastructure teams. Lead the Security on-boarding of Chartable into Spotify's security program and assisted them in increasing their incident response preparedness.
	\item Transitioned SIEM platforms from Splunk to Google Chronicle to better investigate security events and reduce response time. 
	%\item Reorganized the Incident Repsonse Program to better separate responsibilities during an incident which lead to a doubling of participants in the rotation.
	%\item Matured the Detection team's project and work management by leading the adoption of Jira. Worked as Scrum master to teach agile project management. 
\end{itemize}
\begin{tabular}{l|l}
{Feb '18 - May '20} & Senior Security Operations Engineer - MailChimp, Atlanta\\
\end{tabular}
\begin{itemize}[itemsep=0.5pt]
	\item Created and lead the Security Operations team of three people to oversee defensive security in the Operations department.
	\item Lead a year long project to revamp Secure Shell (SSH) access across the company using Okta Advanced Server Access (Okta ASA), an Okta-based zero-trust service. 
	%\item Collaborated with senior engineering leadership to implement secure best practices during Google Cloud Platform (GCP) adoption and migration.
	\item Implemented a security logging, monitoring, and alerting pipeline in GCP using Forseti, Cloud Security Command Center (CSCC), and other GCP managed services.
	\item Built a security information and event management (SIEM) platform using Elasticsearch (ES) Auditbeat to aggregate auditd, authentication, and network logs.
	%\item Monitored for and assessed new software vulnerabilities then collaborated with engineering teams to patch and mitigate.
	%\item Coordinated annual third-party penetration test engagements and lead mitigation efforts across departments based on test report findings.
	%\item Commanded security incident responses and assisted on-call teams on security issues.
\end{itemize}
\begin{tabular}{l|l}
{Jan '15 - Feb '18} & Systems Engineer - MailChimp, Atlanta\\
\end{tabular}
\begin{itemize}[itemsep=0.5pt]
	\item Scaled infrastructure, monitoring and tooling to support a 10 million doubling of customers in just over three years.
	\item Lead a five person team to perform a risk analysis against Spectre/Meltdown, plan the mitigation strategy, and execute it across the entire company.
	%\item Provided on-call support for infrastructure, internal services and public facing sites.
\end{itemize}

%-----Projects-----%
%\section*{Projects}
%\subsubsection*{Okta Advanced Server Access Adoption for SSH Access}
%\begin{itemize}[noitemsep]
%	\item Enumerated shortcomings of existing Secure Shell (SSH) access solution, defined acceptance criteria for a new solution and researched potential solutions.
%	\item Identified Okta Advanced Server Access (Okta ASA), a zero-trust model solution, and evaluated it through a proof-of-concept test.
%	\item Lead the project team to roll ASA out company-wide and then to deprecate the previous solution.
%\end{itemize}
%\subsubsection*{Security Logging and Monitoring}
%\begin{itemize}[noitemsep]
%	\item Implemented security monitoring and alerting in Google Cloud Platform through Forseti and Cloud Security Command Center.
%	%\item Advised senior engineering leadership on security best practices during Google Cloud Platform (GCP) adoption.
%	\item Created an Elasticsearch Security Information and Event Management (SIEM) platform to aggregate security logs from software across the company.
%	\item Used ElastAlert to page and alert on-call teams of potential security incidents from SIEM logs.
%\end{itemize}
%\subsubsection*{Spectre/Meltdown Mitigation}
%\begin{itemize}[noitemsep]
%	\item Performed initial discovery and research of the Spectre/Meltdown vulnerabilities
%	\item Assessed risk to the infrastructure and designed the initial response project
%	\item Lead a five person project team while providing bimonthly updates to Mailchimp senior leadership
%\end{itemize}
%\subsubsection*{Home Lab}
%\begin{itemize}[noitemsep]
%	\item Created and configured a Kubernetes cluster in Google Kubernetes Engine (GKE)
%	\item Utilized the Helm package manager to install a Virtual Private Network (VPN), Hashicorp Vault and a Wordpress website
%	\item Created a repository of bootstrap scripts and Kubernetes manifests to manage the cluster and provide a backup state (https://github.com/jconard3/gcp\_k8)
%\end{itemize}
%\subsubsection*{Intro to Puppet}
%\begin{itemize}[noitemsep]
%	\item Presented a hour long lecture with evolving participatory demo (https://github.com/jconard3/puppet\_demo)
%	\item Introduced concepts of configuration management, idempotence, and declarative languages
%	\item Described puppet manifests, catalog compilation, and architecture in a typical Puppet run
%\end{itemize}
% \subsubsection*{Create a Personal VPN}
% \begin{itemize}[noitemsep]
% 	\item Fifteen minute lightning talk given during a meet-and-greet hosted by the Security team for Cyber Security Awareness Month 2018
% 	\item Showed the basic concepts of a VPN and how they can be useful for personal security
% 	\item Talked to how self-hosted VPNs can be more secure and inexpensive as paid solutions
% \end{itemize}

%-----Skills-----%
\section*{Skills}
\begin{tabular}{rl}
 	Languages: & Go, Python, Bash, Puppet, Terraform, Scala\\
	Platforms: & CentOS, Ubuntu, GCP, Amazon Web Services (AWS)\\
  	Software: & Google Chronicle, Splunk, Elasticsearch (ES), Kubernetes, Github, Jira\\
	Security Tools: & AWS CloudTrail and GuardDuty, GCP CSCC, Capsule8, OSQuery, auditd\\
\end{tabular}

%-----Education-----%
\section*{Education}
\begin{tabular}{rl}
    2009 - 2014 & Bachelor of Science in {Computer Engineering}\\
    & {Georgia Institute of Technology}, Atlanta\\
\end{tabular}
\end{document}